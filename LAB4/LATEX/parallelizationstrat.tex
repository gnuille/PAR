%parallelization strategy
\section{Parallelization strategy}
\justify
After evaluating the task decomposition and choosing which region we are going to parallelize, we have to choose a strategy. Since the approach to the n-queens problem given to us is recursive, we have to choose between the two strategies that have been taught to us; leaf or tree.
\justify
\subsection{Task creation and synchronization}
Looking at the code it's clear that the tree strategy will be more suited to the problem, since the base case doesn't do a huge amount of calculations and each recursive call does parallelizable work (a for-loop and memory copy), so creating tasks as we go through the recursion will increase the performance.
\justify
In the following code you can see our initial approach:

\begin{lstlisting}
void nqueens(int n, int j, char *a) {
    int i;
    if (n == j) {
        if( sol == NULL ) {
            sol = malloc(n * sizeof(char));
            memcpy(sol, a, n * sizeof(char));
        }
        sol_count += 1;
    } else {
        for ( i=0 ; i < n ; i++ ) {
            #pragma omp task
            {
                char * new_a = alloca(n * sizeof(char));
                memcpy(new_a, a, j*sizeof(char));
                new_a[j] = (char) i;
                if (ok(j + 1, new_a)) {
                    nqueens(n, j + 1, new_a);
                }
            }
        }
        #pragma omp taskwait
    }
}
\end{lstlisting}
\justify
We decided to create one task for each iteration of the for-loop and then, to satisfy dependences, synchronize them at the end of the loop with a taskwait. One important change from the initial version is the use of \textbf{\textit{alloca}} and \textbf{\textit{memcpy}}. Each task is reading and writing the memory to check if the possible position of the queen is correct, and each of them has to do so independenly from other tasks, so what we have chosen to do to solve the problem is to copy the vector \textbf{\textit{a}} for each task. This will give us some overhead, as copying memory is not a fast operation.
\justify

\subsection{Race conditions}
Having finished the task creation and synchronization of the code, there is still work to do. The last version of the code doesn't give the correct result yet due to two race conditions. The first one and the easier to solve is the increment of the solution counter variable. This global variable is read and then written by each task, and since two of them can access it at the same time (giving a different outcome depending on the order) we are in front of a race condition problem.
\justify
To solve it, we placed a \textbf{\textit{\#pragma omp atomic}} before the operation so we guarantee that only one thread is doing the increment at a time.
\begin{lstlisting}
#pragma omp atomic
sol_count += 1;
\end{lstlisting}
\justify
With this change, the execution of the code gives the correct answer to the problem regarding the total number of possible solutions. However, it doesn't print the same solution each time and looking at the code, the solution may even we wrong.
\justify
This is due to the second race condition. Two or more threads can enter the \textbf{\textit{if (sol==NULL)}} part of the base case. Not only will this threads be the same each time (since it depends on how fast they traveled the recursive tree) but they can be copying the memory a the same time, giving a mixed solution (some parts of the solution of one thread and some parts of another one).
\justify
The initial fix to this problem is to declare a critical region that protects the whole if statement. This would work fine, but it has some overhead. each thread that gets to the base case will have to synchronize and wait to check the condition, even though we know that once a solution is found it will never be false again. We can optimize it doing a double-check before entering the critical zone : 
\begin{lstlisting}
if( sol == NULL ) {
    #pragma omp critical
    if( sol == NULL ){
        sol = malloc(n * sizeof(char));
        memcpy(sol, a, n * sizeof(char));
    }
}
\end{lstlisting}
\justify
The critical zone mentioned before is now inside another if statement; this one not protected by a critical zone. Doing this only the firsts threads to reach the base case will enter to the critical zone. The other threads will not enter inside the first if statement so they wont execute the critical pragma.
\clearpage
\justify
The final state of the code, in this section, is the following:
\begin{lstlisting}
void nqueens(int n, int j, char *a) {
    int i;
    if (n == j) {
        if( sol == NULL ) {
            #pragma omp critical
            if( sol == NULL ){
                sol = malloc(n * sizeof(char));
                memcpy(sol, a, n * sizeof(char));
            }
        }
        #pragma omp atomic
        sol_count += 1;
    } else {
        for ( i=0 ; i < n ; i++ ) {
            #pragma omp task
            {
                char * new_a = alloca(n * sizeof(char));
                memcpy(new_a, a, j*sizeof(char));
                new_a[j] = (char) i;
                if (ok(j + 1, new_a)) {
                    nqueens(n, j + 1, new_a);
                }
            }
        }
        #pragma omp taskwait
    }
}
\end{lstlisting}

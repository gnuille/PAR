%introduction
\section{Introducion}
\justify
In this labaratory assignement we will be working with the \nq problem.
This problem states that we have to find all possible solutions for
placing \nq in an $N \times N$ chessboard so that each queen can't
attack any of the other queens placed.
\justify
The workarround for this problem is to use a \bab technique. 
This technique tries all possible combinations with a recursive
algorithm using a backtracking aproach.
\justify
Take into account that the computational cost of this problem is expensive 
by reason of it has to compute all combinations in a chess board. 
For example, in an $8 \times 8$ chess board the algorithm tries ${64 \choose 8}$ 
combinations, aproximately \num{4.42e9}.
\justify
Consequently, we will be analysing the potential parallelism of the \nq problem
and proposing a parallel version of the code, hopefully improving it's 
performance.
